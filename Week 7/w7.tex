% ---
\documentclass{article}


\usepackage[USenglish]{babel} 	% Change hyphenation rules
\usepackage{hyperref} 				% Add a link to your document
\usepackage{graphicx}				% Add pictures to your document
\graphicspath{ {./images/} }	% image directory
\usepackage{listings} 				% Source code formatting and highlighting
\lstset{basicstyle=\ttfamily}		%Typewriter font for code writing
\usepackage{geometry}
\geometry{margin=1in}
\usepackage{enumitem}
\usepackage{float}
\usepackage{amsmath, amssymb}
%\floatstyle{boxed}
\restylefloat{figure}
\usepackage{mathabx}
\usepackage{fancyhdr}
\usepackage{mdframed}
\usepackage{needspace}
\usepackage[dvipsnames]{xcolor}
\usepackage{scrextend}

% Colors
\definecolor{blu}{rgb}{0,0,1}
\def\blu#1{{\color{blu}#1}}
\definecolor{gre}{rgb}{0,.5,0}
\def\gre#1{{\color{gre}#1}}
\definecolor{red}{rgb}{1,0,0}
\def\red#1{{\color{red}#1}}
\def\norm#1{\|#1\|}

\pagestyle{fancy}
\fancyhf{}
\lhead{\bf CPSC 322 \\ Week 7 }
\rhead{\bf Jeremi Do Dinh \\ 61985628}
\rfoot{Page \thepage}



\usepackage{tikz}						% Graph drawing tools
\usetikzlibrary {positioning}

\usepackage{breqn}
\usepackage{multicol} 				% Multiple column functionality
\usepackage{blindtext}
\newcommand{\centerfig}[2]{\begin{center}\includegraphics[width=#1\textwidth]{#2}\end{center}}

\newmdenv[
topline=false,
bottomline=false,
skipabove=\topsep,
skipbelow=\topsep
]{siderules}

\begin{document}
\noindent \url{https://www.cs.ubc.ca/~jordon/teaching/cpsc322/2019w2/}
\section*{Lecture 18 - Logic: Intro \& Syntax}
\textbf{February 24th, 2020}\\
\url{https://www.cs.ubc.ca/~jordon/teaching/cpsc322/2019w2/lectures/lecture18.pdf}\\
\\
\textbf{Logic is the language of Mathematics}, used to define formal structures (e.g., sets, graphs) and to prove statements about them.\\
\\
Consider how to represent an environment about which we have only partial (but certain) information. What do we need to represent?:
\begin{itemize}
	\item \textbf{\blu{Facts}} about the environment
\item \textbf{{\color{Fuchsia}Rules}} of how the environment works
\end{itemize}
\subsection*{Propositional Logic}
\textbf{We begin with propositional logics}, since this is the starting point for more complex logics. The primitive elements of propositional logic are \textbf{\textit{propositions}}: Boolean variables that can be  $ \{true, false\} $.\\
\\
The goal is to illustrate the basic ideas as a starting point for more complex logics. Boolean nature can be exploited for efficiency.\\
\\
The proposition symbols $ p_1, p_2, \dots $ etc. are sentences:
\begin{itemize}
	\item If $ S $ is a sentence, then $ \lnot S $ is a sentence (\blu{negation})
	\item If $ S_1 $ and $ S_2 $ are sentences, then $ S_1 \land S_2 $ is a sentence (\blu{conjunction})
	\item If $ S_1 $ and $ S_2 $ are sentences, then $ S_1 \lor S_2 $ is a sentence (\blu{disjunction})
	\item If $ S_1 $ and $ S_2 $ are sentences, then $ S_1 \Rightarrow S_2 $ is a sentence (\blu{implication})
	\item If $ S_1 $ and $ S_2 $ are sentences, then $ S_1 \Leftrightarrow S_2 $ is a sentence (\blu{biconditional})
\end{itemize}
\subsubsection*{Propositional Logics in practice}
\begin{itemize}
	\item Agent is told (perceives) some facts about the world (\blu{a set of true propositions})
	\item Agent is told (already knows / learns) how the world works ({\color{Fuchsia}a set of logical formulas})
	\item Agent can answer \gre{yes/no questions} about whether other facts \textbf{\blu{must be true}}
\end{itemize}
\subsubsection*{Using Logics to make inferences...}
\begin{enumerate}
	\item  Begin with a task domain.
	\item Distinguish those things you want to talk about
	\item Choose symbols in the computer to denote propositions
	\item Tell the system knowledge about the domain
	\item Ask the system whether new statements about the domain are true or false
\end{enumerate}

\subsection*{Propositional Definite Clauses}
This is our first representation and reasoning system. We only have two kinds of statements:
\begin{itemize}
	\item that a \textbf{proposition is true}: $ p_2 $
	\item that a proposition is true if one or more other propositions are true $ p_2 \leftarrow p_1 \land p_3 $
\end{itemize}
\subsection*{Propositional Definite Clauses: Syntax}
\textbf{Def: (atom)} \\
An \blu{\textbf{atom}} is a symbol starting with a lower case letter. \\
\\
\textbf{Def: (body)}\\
A \textbf{\blu{body}} is an atom, or is of the form $ b_1 \land b_2 $ where $ b_1 $ and $ b_2 $ are bodies. \\
\\
\textbf{Def: (definite clause)}\\
A \textbf{\blu{definite clause}} is an atom or is a rule of the form $ h \leftarrow b $ where $ h $ is an atom and $ b $ is a body. We read this as "$ h $ if $ b $".\\
\\
\textbf{Def: (knowledge base)}\\
A \textbf{\blu{knowledge base}} is a set of definite clauses.

\newpage

\section*{Lecture 19 - Logic: Semantics and Bottom-Up Proofs}
\textbf{February 26th, 2020}\\
\url{https://www.cs.ubc.ca/~jordon/teaching/cpsc322/2019w2/lectures/lecture19.pdf}\\
\subsection*{Propositional Definite Clauses Semantics: Interpretation}
Semantics allows you to relate the symbols in the logic to the domain you're trying to model. An atom can be $ T $ or $ F $. \\
\\
\textbf{Def: (Interpretation)}\\
An \textit{\blu{interpretation}} $ I $ assigns a truth value to each atom. \\
\\
So an interpretation is just a \blu{\textbf{possible world}}\\
\\
We can use the \textbf{interpretation} to determine the truth value of \textbf{clauses} and \textbf{knowledge bases}. \\
\\
\textbf{Def: (truth values of statements)}\\
A \blu{body $ b_1 \land b_2 $ is true in $ I $} if and only if $ b_1 $ is true in $ I $ and $ b_2 $ is true in $ I $. \\
\\
\textbf{Def: (truth values of statements cont')}\\
A \blu{rule $ h \leftarrow b $ is false in $ I $} if and only if \blu{$ b $ is true in $ I $} and \blu{$ h $ is false in $ I $}. \\
\\
\textbf{Semantics of a KB}: \\
A \textbf{\red{knowledge base KB}} is true in $ I $ if and only if every clause in KB is true in $ I $. \\
\\
\textbf{Def: Model}\\
A \textbf{model} of a set of clauses (a Knowledge base) is an interpretation in which all clauses are \textbf{\textit{true}}.

\subsection*{Logical Consequence}
\textbf{Def: (logical consequence)}\\
If $ KB $ is a set of clauses and $ G $ is a conjunction of atoms, then $ G $ is a logical consequence of $ KB $, written a $ KB \vDash G $, if $ G $ is true in every model of $ KB $.\\
\begin{itemize}
	\item It can also be said that $ G $ \blu{logically follows} from $ KB $, or that $ KB $ entails $ G $. 
	\item In other words, $ KB \vDash G $ if there is no interpretation $ I $ in which $ KB $ is $ true $ and $ G $ is $ false $.
\end{itemize}

\subsection*{Soundness and Completeness}
Suppose we are told that we have a \textbf{\blu{proof procedure for PDCL}}. What needs to be shown in order for us to trust the given procedure? \\
\\
\textbf{Recall:} \\
$ K \vdash G $ means that $ G $ can be derived by, the given proof procedure, from $ KB $.\\
$ K \vDash G $ means that $ G $ is true in all models of $ KB $\\
\\
\underline{\textbf{Def: (soundness)}}\\
A proof system is \blu{sound} if $ KB \vdash G $ implies that $ KB \vDash G $\\
\\
$\underline{\textbf{Def: (completeness)}}$\\
A proof procedure is \blu{complete} if $ KB \vDash G $ implies that $ KB \vdash G $


\subsection*{Bottom-up Ground Proof Procedure}
One mainly used \blu{rule of derivation}, which is a generalized form of \textbf{\textit{modus ponens}} is defined as follows:\\
\\
If $ h \leftarrow b_1 \land \dots \land b_m $ is a clause in the knowledge base, and each $ b_i $ has been derived, then $ h $ can be derived. \\
\\
The procedure uses the above rule to find whether $ G $ logically follows from the knowledge base. We then have that \blu{$ KB \vdash G $ if at the end of the procedure $ G \subseteq C $}:
\begin{siderules}
	\textbf{Algorithm:}
\begin{itemize}[label = ]
\item $ C := \{\} $
\item \textbf{repeat}:
\begin{itemize}[label = ]
	\item 	\textbf{select} clause "$ h \leftarrow b_1 \land ... \land b_m $" in KB such that $ b_i \in C $ for all $ i $, and $ h \notin C $. 
	\item $ C := C \cup \{h\} $
\end{itemize}
\item \textbf{until} no more clauses can be selected
\end{itemize}
\end{siderules}


\end{document}