% ---
\documentclass{article}


% Packages
% ---
\usepackage{amsmath,amssymb,amsthm} 	% Advanced math typesetting
% \usepackage[utf8]{inputenc} 	% Unicode support (Umlauts etc.)
\usepackage[USenglish]{babel} 	% Change hyphenation rules
\usepackage{hyperref} 				% Add a link to your document
\usepackage{graphicx}				% Add pictures to your document
\graphicspath{ {./images/} }	% image directory
\usepackage{listings} 				% Source code formatting and highlighting
\lstset{basicstyle=\ttfamily}		%Typewriter font for code writing
\usepackage{geometry}
\geometry{margin=1in}
\usepackage{enumitem}
\usepackage{float}
%\floatstyle{boxed}
\restylefloat{figure}
\usepackage{mathabx}
\usepackage{fancyhdr}
\usepackage[dvipsnames]{xcolor}
\usepackage{scrextend}
\usepackage{cancel}
\usepackage{algorithmicx}

% Colors
\definecolor{blu}{rgb}{0,0,1}
\def\blu#1{{\color{blu}#1}}
\definecolor{gre}{rgb}{0,.5,0}
\def\gre#1{{\color{gre}#1}}
\definecolor{red}{rgb}{1,0,0}
\def\red#1{{\color{red}#1}}
\def\norm#1{\|#1\|}

\theoremstyle{definition}
\newtheorem{definition}{Def}

\pagestyle{fancy}
\fancyhf{}
\lhead{\bf CPSC 322 \\ Week 4 }
\rhead{\bf Jeremi Do Dinh \\ 61985628}
\rfoot{Page \thepage}



\usepackage{tikz}						% Graph drawing tools
\usetikzlibrary {positioning}

\usepackage{breqn}
\usepackage{multicol} 				% Multiple column functionality
\usepackage{blindtext}
\newcommand{\centerfig}[2]{\begin{center}\includegraphics[width=#1\textwidth]{#2}\end{center}}


\begin{document}

\noindent \url{https://www.cs.ubc.ca/~jordon/teaching/cpsc322/2019w2/}

\section*{Lecture 10: CSPs - Introduction}
\textbf{January 28th, 2020} \\
\url{https://www.cs.ubc.ca/~jordon/teaching/cpsc322/2019w2/lectures/lecture10.pdf}


\subsection*{Variables/Features, domains and Possible Worlds}
Variables/Features:
\begin{itemize}
	\item We denote variables using capital letters
	\item Each variable $ V $ has a \blu{domain} of possible values, denoted $ dom(V) $. 
	\item Variables can be of several main kinds:
	\begin{itemize}
		\item \blu{Boolean}: $ |dom(V)| = 2 $
		\item \blu{Finite}: the domain contains a finite number of values
		\item \blu{Infinite but discrete}: the domain is countably infinite
		\item \blu{Continuous}: e.g.,real numbers between 0 and 1
	\end{itemize}
\end{itemize}
A possible world is a complete assignment of values from domains to variables. 

\begingroup
\leftskip 2em
\subsubsection*{Examples:}
\begin{itemize}[label=$ \rightarrow $]
	\item Crossword puzzle:
	\centerfig{0.5}{crossword}
	\begin{itemize}
		\item variables are words that have to be filled in (63 in this case)
		\item domains are valid English words of required length
		\item possible worlds: all ways of assigning words
	\end{itemize}
\item Crossword 2:
\begin{itemize}
	\item variables are cells (individual squares) in the 15x15 grid
\item domains are letters of the alphabet
\item possible worlds: all ways of assigning letters to cells
\end{itemize}
\item Sudoku
\begin{itemize}
	\item variables are empty cells
	\item domains are numbers between 1 and 9
	\item possible worlds: all ways of assigning numbers to cells
\end{itemize}
\item n-Queens problem
\begin{itemize}
	\item variable: location of a queen on a chess board
	\begin{itemize}
		\item there are n of them in total, hence the name
	\end{itemize} 
	\item domains: grid coordinates ($ n^2 $)
	\item possible worlds: locations of all queens
	\centerfig{0.25}{queens}
	
\end{itemize}



\item Scheduling problem: \\
In the scheduling problem we have the following definition of the sets in question:
\begin{itemize}
	\item The \blu{variables} are the different tasks that need to be scheduled (e.g., course in a university; job in a machine shop)
	\item The \blu{domains} are the different combinations of times and locations for each task (e.g., time/room for course; time/machine for job)
	\item The \blu{possible worlds} are a time/location assignment to each task
\end{itemize}
\end{itemize}
\endgroup

\subsection*{Constraints}
Constraints are restrictions on the values that one or more variables can take: \\\\
\textbf{Def:} \blu{A unary constraint} is a restriction involving only one variable. \\\\
\textbf{Def:} \blu{A k-ary constraint} is a restriction involving the domains of k different variables (However all k-ary constraints can be represented by binary constraints) \\ \\
\textbf{Constraints can be specified by:}
\begin{itemize}
	\item Giving a function that returns true when given values for each variable satisfy the constraint
	\item Giving a list of valid domain values for each variable participating in the constraint
\end{itemize}
\textbf{Def:} A possible world satisfies a set of constraints if the set of variables involved in each constraint take values that are consistent with that constraint. \\ \\
Constraints in the above examples:
\begin{itemize}
	\item Crossword Puzzle: \\
	\textit{Constraints}: words have the same letters at points where they intersect
	\item Crossword 2: \\
	\textit{Constraints}: sequences of letters form valid English words
	\item Sudoku: \\
	\textit{constraints}: rows, columns, boxes contain all different numbers
	\item n-Queens problem \\
	\textit{constraints}: no queen can attack another
	\item Scheduling Problem \\
	\textit{constraints:}
	\begin{itemize}
		\item tasks can't be scheduled in the same location at the same time
		\item certain tasks can be scheduled only in certain locations
		\item some tasks must come earlier than others; etc.
	\end{itemize}
	
\end{itemize}


\subsection*{CSPs}
\textbf{Def: (Constraint Satisfaction Problem)} \\ 
A constraint satisfaction problem consists of
\begin{itemize}
	\item a set of variables
\item a domain for each variable
\item a set of constraints
\end{itemize}
\textbf{Def: (model / solution)} \\
A \blu{model} of a CSP is an assignment of values to variables (i.e. a \textbf{possible world}) that satisfies all of the constraints.


\subsubsection*{Variants}
We may want to solve the following problems using a CSP:
\begin{enumerate}[label = \Alph*) ]
	\item determine whether or not a model \blu{exists}
	\item \blu{find} a model
	\item \blu{find all} of the models
	\item \red{count the number of the models}
	\item find the \blu{best} model given some model quality 
	\begin{itemize}
		\item this is now an optimization problem
	\end{itemize}
	\item \red{determine whether some properties of the variables hold in all models}
\end{enumerate}

\textbf{To summarize:}
\begin{itemize}
	\item Need to think of search beyond simple goal driven planning agent.
\item We started exploring the first AI Representation and Reasoning framework: \textbf{CSPs}

\end{itemize}

\newpage

\section*{Lecture 11: CSPs - Search and Arc Consistency}
\textbf{January 28th \& 30th, 2020} \\
\url{https://www.cs.ubc.ca/~jordon/teaching/cpsc322/2019w2/lectures/lecture11.pdf}

\subsection*{Generate-and-Test algorithm}
This algorithm focuses on generating all of the possible worlds at a single time, and test them to see if any violate any constraints. The pseudocode is as follows:
\begin{lstlisting}[tabsize=3]
		for a in domA 
			for b in domB
				for c in domC
					if (abc) satisfies all constraints:
						return (abc)
		return NULL
\end{lstlisting}

\subsection*{CSPs as Search Problems}
We define CSPs the following way:
\begin{itemize}
	\item \textbf{\blu{states}}: assignments of values to a subset of the variables
	\item \textbf{\blu{start}} state: the empty assignment (no variables assigned values)
	\item \textbf{\blu{neighbours}} of a state: nodes in which values are assigned to one additional variable
	\item \textbf{\blu{goal state}}: a state which assigns a value to each variable, and satisfies all of the constraints
\end{itemize}
For a CSP problem, some search strategies will work better than others. It is important to notice that for a problem with $ n $ variables all of the solutions are at depth $ n $, thus there is \textsl{NO} role for a heuristic function. \\ \\
Also, since the search space is finite, and without cycles, DFS is the best strategy for finding a solution. \\ \\
We can avoid exploring some sub-trees i.e. prune the DFS Search tree, by implementing the following strategy:
\begin{itemize}
	\item once we consider a path whose end node violates one or more constraints, we know that \textbf{\blu{a solution cannot exist below that point}}
	\item thus we should \textbf{\blu{remove that path}} rather than continuing to search
\end{itemize}
The followign is an example, where the variables set is $ \{A, B, C\} $, the domains for each variable is: $ \{1, 2, 3, 4\} $ and the constraints are: $ \{ A < B, B<C\} $:
\centerfig{0.6}{ex_csp1}
At this point it is important to note that the algorithm's efficiency will depend on the order in which the nodes are expanded. \\ \\
Consequently we ask whether it is possible to do better than search. 
\subsection*{Consistency}
\textbf{Key Ideas}: prune the domains as much as possible before “searching” for a solution. \\ \\
\textbf{Def:} A variable is domain consistent if no value of its domain is ruled impossible by any unary constraints. \\ \\
We therefore need a way to deal with constraints that involve more than one variable. For this we define constraint networks. 

\subsection*{Constraint Networks}
\textbf{Def: (Constraint Network)} \\
A \blu{constraint network} is defined by a graph, with:
\begin{itemize}
	\item one \textbf{node} for every \textbf{variable}
	\item one \textbf{node} for every \textbf{constraint}
\end{itemize}
and undirected edges running between variable nodes and constraint nodes whenever a given variable is involved in a given constraint

\subsection*{Arc Consistency}
\textbf{Def: (Arc Consistency)} \\
An arc $ \langle X,r(X,Y) \rangle $ is arc consistent if for each value $ x $ in $ dom(X) $ there is some value $ y $ in $ dom(Y ) $ such that $ r(x, y) $ is satisfied.


\subsubsection*{How can we enforce Arc Consistency?}
If an arc $ \langle X,r(X,Y) \rangle $ is not arc consistent, all values $ x $ in $ dom(X) $ for which there is no corresponding value in $ dom(Y ) $ may be deleted from $ dom(X ) $ to make the arc $ \langle X,r(X,Y) \rangle $ consistent. This removal will never rule out any models/solutions.  \\ \\
\textbf{A network is arc consistent if all arcs are consistent. }


\newpage

\section*{Lecture 12: CSPs - Arc Consistency \& Domain Splitting}
\textbf{January 30th, 2020} \\
\url{https://www.cs.ubc.ca/~jordon/teaching/cpsc322/2019w2/lectures/lecture12.pdf}

\subsection*{Arc Consistency Algorithm: high level strategy}
The idea is to consider the arcs in turn, making each arc consistent. However some may be revisited. Notable, when we reduce the domain of a variable X to make an arc $ \langle X, c \rangle $ arc consistent, we add every arc $ \langle Z, c' \rangle $ where $ c' $ involves $ Z $ and $ X $:
\centerfig{0.8}{arc_consistency}
It is not necessary to add other arcs $ \langle X, c' \rangle $ where $ c \ne c' $, since if an arc $ \langle X, c' \rangle $ was consistent before, it will still be arc consistent (in the "for all" we'll just check fewer values)

\subsubsection*{Arc Consistency Pseudocode}
\begin{lstlisting}[tabsize=3]
	TDA <- all arcs in constraint network 
	while TDA is not empty:
		select arc a from TDA
		if a is not consistent:
			make a consistent
			add arcs to TDA that may now be inconsistent
\end{lstlisting}
A higher level version of the algorithm is shown below:

\begin{lstlisting}[tabsize=3]
	Procedure GAC(V,dom,C)
		Inputs:
			V: a set of variables
			dom: a function such that dom(X) is the domain of variable X
			C: set of constraints to be satisfied
		Output:
			arc-consistent domains for each variable
		Local
			D_X is a set of values for each variable X 
			TDA is a set of arcs
		for each variable X do:
			D_X <- dom(X)
			TDA <- {<X, c> | X in V, c in C and X in scope(c)}
		while TDA is not empty:
			select <X, c> in TDA
			TDA <- TDA.remove(<X, c>)
			ND_X <- {x | x in D_X and there exists y in D_Y s.t. (x,y) satisfies c}
			if (ND_X != D_X) then:
				TDA <- TDA + {<Z, c'> | X in scope(c') and c' != c \
						and Z in scope(c') - {X}} 
				D_X <- ND_X
		return {D_X | X is a variable}
\end{lstlisting}

\subsubsection*{Arc Consistency Algorithm Complexity}
Let’s determine Worst-case complexity of this procedure:
\begin{itemize}
	\item Let the \blu{max size of a variable domain} be $ d $
	\item Let the \blu{number of variables} be $ n $
	\item Let all constraints be \textbf{\blu{binary}}
\end{itemize}
In this case we have that:
\begin{itemize}
	\item The maximum number of binary constraints is: $ \frac{n(n-1)}{2} $
	\item The number times the same arc can be inserted in back into the ToDoArc list is $ d $
	\item The number of steps involved in checking the consistency of an arc is: $ d^2 $
\end{itemize}
Consequently, the time complexity of the Arc Consistency algorithm is $ O(n^2d^3) $

\subsubsection*{Interpreting Solutions}
There are three possible outcomes to the algorithm, and they are interpreted as follows:
\begin{itemize}
	\item One domain is empty $ \rightarrow $ there is no solution
	\item Each domain has a unique value $\rightarrow$ there is a unique solution
	\item Some domains have more than one value $\rightarrow$ \red{zero or more solutions}:
	\begin{itemize}
		\item in this case, arc consistency isn't enough to solve the problem: we still need to perform \textit{search}
	\end{itemize}
\end{itemize}
\subsection*{Domain Splitting}
When arc consistency ends some domains have more than one value, which means there may or may not be a solution... We have two options:
\begin{enumerate}[label = \Alph* )]
	\item Apply Depth-First Search with Pruning
	\item Split the problem in a number of (eg. two) disjoint cases
\end{enumerate}
By reducing dom(X) we may be able to run AC again, since we simplify the problem using \textbf{arc consistency}:
\begin{lstlisting}[tabsize=3]
	If there is no unique solution (dom(X)>1 for at least one variable):
		Split X
		For all the splits:
			Restart arc consistency on arcs <Z, r(Z,X)>
\end{lstlisting}
These are the arcs that are possibly inconsistent.  \\ \\
The disadvantage of this is that we are required to keep all of the CSPs around rather than smaller simpler states of DFS. 
\subsubsection*{Searching by Domain splitting}
\centerfig{0.75}{domainsplitting}
More formally: Arc consistency with domain splitting as another formulation of CSP as search:

\begin{lstlisting}[tabsize = 3]
Start state: run AC on vector of original domains (dom(V_1), ..., dom(V_n))

States: "remaining" domains (D(V_1), ..., D(V_n)) for the vars with D(V_i)\
		in dom(V_i) for each V_i
		
Successor function: split one of the domains + run arc consistency

Goal state: vector of unary domains that satisfies all constraints

Solution: any goal state
\end{lstlisting}




\end{document}