% ---
\documentclass{article}


\usepackage[USenglish]{babel} 	% Change hyphenation rules
\usepackage{hyperref} 				% Add a link to your document
\usepackage{graphicx}				% Add pictures to your document
\graphicspath{ {./images/} }	% image directory
\usepackage{listings} 				% Source code formatting and highlighting
\lstset{basicstyle=\ttfamily}		%Typewriter font for code writing
\usepackage{geometry}
\geometry{margin=1in}
\usepackage{enumitem}
\usepackage{float}
\usepackage{amsmath, amssymb}
%\floatstyle{boxed}
\restylefloat{figure}
\usepackage{mathabx}
\usepackage{fancyhdr}
\usepackage{mdframed}
\usepackage{needspace}
\usepackage[dvipsnames]{xcolor}
\usepackage{scrextend}

% Colors
\definecolor{blu}{rgb}{0,0,1}
\def\blu#1{{\color{blu}#1}}
\definecolor{gre}{rgb}{0,.5,0}
\def\gre#1{{\color{gre}#1}}
\definecolor{red}{rgb}{1,0,0}
\def\red#1{{\color{red}#1}}
\def\norm#1{\|#1\|}

\pagestyle{fancy}
\fancyhf{}
\lhead{\bf CPSC 322 \\ Week 8 }
\rhead{\bf Jeremi Do Dinh \\ 61985628}
\rfoot{Page \thepage}



\usepackage{tikz}						% Graph drawing tools
\usetikzlibrary {positioning}

\usepackage{breqn}
\usepackage{multicol} 				% Multiple column functionality
\usepackage{blindtext}
\newcommand{\centerfig}[2]{\begin{center}\includegraphics[width=#1\textwidth]{#2}\end{center}}

\newmdenv[
topline=false,
bottomline=false,
skipabove=\topsep,
skipbelow=\topsep
]{siderules}

\begin{document}
\noindent \url{https://www.cs.ubc.ca/~jordon/teaching/cpsc322/2019w2/}
\section*{Lecture 20 - Bottom Up proof: soundness and completeness}
\textbf{March 3rd, 2020}\\
\url{https://www.cs.ubc.ca/~jordon/teaching/cpsc322/2019w2/lectures/lecture20.pdf}
\subsubsection*{Goals:}
\begin{itemize}
	\item Prove that the bottom up proof system is \textbf{\blu{sound}}
	\item Prove that the bottom up proof system in \textbf{\blu{complete}}
\end{itemize}
\textbf{Recap of soundness and completeness}\\
\\
\textbf{Def: (Generic soundness of a proof procedure)}\\
If $ G  $ can be proven by the procedure ($ KB \vdash G $) then $ G $ is logically entailed by the $ KB $ ($ KB \vDash G $)\\
\\
\textbf{Def: Generic completeness of proof procedure}\\
If $ G $ is logically entailed by the $ KB $ ($ KB \vDash G $), then $ G $ can be proven by the procedure ($KB \vdash G $). \\
\\
In other words:
\begin{itemize}
	\item Everything \gre{\textbf{derived}} from a \textbf{\blu{sound}} proof procedure is \textbf{\color{Fuchsia}entailed} by the KB.
	\item Everything \textbf{\color{Fuchsia}entailed} by the KB can be \gre{\textbf{derived}} a \textbf{\blu{complete}} proof procedure
\end{itemize}
We had the algorithm for computing the set of consequences of $ KB $:
\begin{itemize}[label = ]
	\item $ C := \{\} $
	\item \textbf{repeat}:
	\begin{itemize}[label = ]
		\item 	\textbf{select} clause "$ h \leftarrow b_1 \land ... \land b_m $" in KB such that $ b_i \in C $ for all $ i $, and $ h \notin C $. 
		\item $ C := C \cup \{h\} $
	\end{itemize}
	\item \textbf{until} no more clauses can be selected
\end{itemize}
So BU is sound if all of the atoms in $ C $ are logically entailed by KB
\subsubsection*{Soundness}
Every atom in $ C $ is a logical consequence of $ KB $.\\
\\
\textbf{Proof (by contradiction):}\\
Suppose there exists an atom in $ C $ that is not a logical consequence of $ KB $. If this is the case, let $ h $ be the first atom added to $ C $ that is not a logical consequence of $ KB $. Let $ I $ be a model in which $ h $ is false.\\\\
Because $ h $ has been generated, there must be some definite clause of the form $ h \leftarrow a_1 \land \dots \land a_m $, such that $ a_1, \dots, a_m $ are all in $ C $.\\
\\
Because $ h $ is the first atom added to $ C $ that is not true in all models of $ KB $, then all the $ a_i $ are generated before $ h $ are true in $ I $. Thus it is a clause where the head is false but the body is true, and thus by the definition of truth clauses this clause is false in $ I $. This is a contradiction to the fact that $ I $ is a model of $ KB $. Thus every element of $ C $ is a logical consequence of $ KB $.

\subsubsection*{Completeness}
If $ G $ is logically entailed by the $ KB $ ($ KB \vDash G $) then $ G $ can be proved by the procedure ($ KB \vdash G $)\\
\\
\textbf{Proof}:\\
Suppose that $ KB \vDash G $. Then $ G $ is true in all models of $ KB $. Thus $ G $ is true in any particular model of $ KB $. \\
\\
We will define a particular model such that if $ G $ is true in that model, $ G $ is proven by the bottom up algorithm. We will therefore define a particular interpretation $ I $ such that iff $ G $ is true in $ I $, $ G $ is proved by the bottom-up algorithm. We will then show that $ I $ is a model. Thus we will define $ I $ such that if $ G $ is true in $ I $, then $ G \subseteq C $.\\
\\
Let $ I $ be an interpretation where each element of $ C $ is \blu{\textbf{true}}, and every other atom is \blu{\textbf{false}}. We claim that $ I $ is a model of $ KB $ (which we'll call the minimal model)
\begin{siderules}
\textbf{Proof} (of claim):\\
\textbf{Assume} that $ I $ is not a model of $ KB $. \textbf{Then} there must exist a clause $ h \leftarrow b_1 \land \dots \land b_m $ in $ KB $ (having zero or more $ b_i $'s) which is \textbf{false} in $ I $. The only way this an occur is if all of the $ b_i $'s are true in $ I $ (are in $ C $) and $ h $ is false in $ I $ (not in $ C $).\\
\\
But if each $ b_i $ belonged to $ C $, bottom up would've added $ h $ to $ C $ as well. Therefore there can be no clause in $ KB $ that is false in interpretation $ I $ (which implies the claim)
\end{siderules}



\newpage
\section*{Lecture 21 - Domain Modeling and Top-Down Proofs}
\textbf{March 3rd \& 5th, 2020}\\
\url{https://www.cs.ubc.ca/~jordon/teaching/cpsc322/2019w2/lectures/lecture21.pdf}
\subsection*{Top-Down Proof Procedure}
\subsubsection*{Bottom-up vs. Top-down}
Bottom up starts with a knowledge base, and derives the set of consequences. 
\centerfig{0.5}{bottom-up-1}
We have that $ G $ is proved, if at the end $ G \subseteq C $. Therefore BU looks at the query only at the end. \\
\\
The \textbf{key idea} of the top down proof system is to \textbf{search backwards} from a query $ G $ to determine if it can be derived from $ KB $. 
\centerfig{0.5}{top-down-1}
TD performs a backwards search, starting at $ G $.
\subsection*{Top-Down Proof Procedure: Elements}
\textbf{Notation:} An \blu{answer clause} is of the form:
\begin{align*}
yes \leftarrow a_1 \land a_2 \land \dots \land a_m
\end{align*}
\textbf{Express query} as an \blu{answer clause} (e.g.: if query = $ a_1 \land \dots \land a_m $), then this yields:
\begin{align*}
yes \leftarrow a_1 \land a_2 \land \dots \land a_m
\end{align*}
\textbf{Rule of inference:} (called SLD Resolution). Given an answer clause of the form
\begin{align*}
yes \leftarrow a_1 \land a_2 \land \dots \land a_m
\end{align*}
and the KB clause:
\begin{align*}
a_i \leftarrow b_1 \land b_2 \land \dots \land b_p
\end{align*}
You can generate the answer clause:
\begin{align*}
yes \leftarrow a_1 \land \dots \land a_{i-1} \land \blu{b_1 \land b_2 \land \dots \land b_p} \land a_{i+1} \land \dots \land a_m
\end{align*}
Some examples are illustrated in the following table:
\begin{center}
\begin{tabular}{| c | c | c |}
	\hline
	answer clause & $ KB $ clause & resulting inference \\
	\hline
	$ yes \leftarrow b \land c $ & $ b \leftarrow k \land f $ & $ yes \leftarrow k \land f \land c $\\
	\hline
	$ yes \leftarrow e \land f $ & $ e $ ($ e \leftarrow $) & $ yes \leftarrow f $\\
	 \hline
\end{tabular}
\end{center}

\subsection*{(Successful) Derivations}
An \blu{answer} is an answer clause with $ m = 0 $. That is, it is the \textbf{empty} answer clause "$ yes \leftarrow $"\\
\\
A (successful) \blu{derivation} of query "$ ?q_1\land \dots \land q_k $" from the $ KB $ is a sequence of answer clauses $ Y_0, Y_1, \dots, Y_n $ such that:
\begin{itemize}
	\item $ Y_0 $ is the answer clause $ yes \leftarrow q_1\land \dots \land q_k $
	\item $ Y_i $ is obtained by \blu{resolving} $ Y_{i-1} $ with a clause in $ KB $
	\item $ Y_n $ is the empty clause
\end{itemize}

\newpage
\section*{Lecture 22 - TD as search, Datalog (variables)}
\textbf{March 5th, 2020}\\
\url{https://www.cs.ubc.ca/~jordon/teaching/cpsc322/2019w2/lectures/lecture22.pdf}
\subsection*{Top Down proof formulated as a search problem}
We define the top down proof system as a search problem in the following way:
\begin{itemize}
	\item \textbf{\blu{State}} is an answer clause
	\item \blu{\textbf{Successor function}} states resulting from substituting one atom with all the clauses of which it is the head.
	\item  \textbf{\blu{Goal state}} empty answer clause
	\item \textbf{\blu{Solution}} start state
	\item \blu{\textbf{Heuristic function}} $ \dots $
\end{itemize}
\subsubsection*{Search Graph}
The following is what a search graph may look when we want to solve $ yes \leftarrow a \land d $, given the following $ KB $:
\centerfig{0.8}{top-down-2}
A \textbf{possible heuristic} could be the \textbf{number of atoms} in the answer clause. \\
\\
However, we may also know that if the body of the answer clause contains a symbol that is not the head of \textit{any} clause in the $ KB $, then we know that the \blu{\textbf{most informative heuristic value}} is \textbf{\textit{zero}}.

\subsection*{Datalog}
\subsubsection*{Representation and Reasoning in Complex domains}
\begin{multicols}{2}
\noindent In complex domains, expressing knowledge with \textbf{propositions} can be quite limiting:\\
\begin{math}
up\_s_2\\
up\_s_3\\
ok\_cb_1\\
ok\_cb_2\\
live\_w_1\\
connected\_w_1\_w_2
\end{math}\\
\noindent It is therefore often natural to consider \textbf{individuals} and their \textbf{properties}.\\
\begin{math}
up(s_2)\\
up(s_3)\\
ok(cb_1)\\
ok(cb_2)\\
live(w_1)\\
connected(w_1, w_2)\\\\
\end{math}
\end{multicols}
By using propositions we have no notion that for instance $ up\_s_2 $ and $ up\_s_3 $ share a meaning, or that $ live\_w_1 $ and $ connected\_w_1\_w_2 $ are about the same individual. 
\subsubsection*{Consequently...}
... we turn \textbf{\color{Fuchsia}propositions} into \textbf{\gre{relations}} that are applied to individuals. We gain much by doing so. Notably:
\begin{itemize}
	\item We can express knowledge that holds for a set of individuals (by introducing \blu{variables}):
	\begin{align*}
	live(W) \leftarrow connected\_to(W, W_1) \land live(W_1) \land wire(W) \land wire(W_1)
	\end{align*}
	\item We can \textbf{ask generic queries}:
	\begin{align*}
	? \quad connected\_to(W, w_1)
	\end{align*}
\end{itemize}
The following slide illustrates the relation between then different proof systems:
\centerfig{0.8}{proof-sys-1}
\subsection*{Datalog: a relational rule language}
Datalog expands the syntax of PDCL(Propositional Definite clause logic). We have the following definitions, where a symbol (or word) is a sequence of letters, digits or an underscore \_:\\
\\
\textbf{Def:} A \red{variable} is a symbol starting with an upper case letter or with $ \_ $ (Examples: $ X, Y $) \\
\\
\textbf{Def:} A \red{constant} is a symbol starting with a lower case letter or a sequence of digits, or is a number constant or a string (Examples: $ alan, w_1 $). \\
\\
\textbf{Def:} A \red{term} is either a variable or a constant (Example: $ X, Y, alan, w_1 $)\\
\\
\textbf{Def:} A \red{predicate symbol} is a symbol starting with a lower case letter. Constants and predicate symbols are distinguishable by their context in the knowledge base (Example: $ live, part-of, connected, in $). \\
\\
\textbf{Def:} An \red{atom} is a symbol of the form \blu{$ p $} or \blu{$ p(t_1, \dots, t_n) $} where \blu{$ p $} is a proposition or predicate symbol, and \blu{$ t_i $} are terms. Each $ t_i $ is referred to as an argument to the predicate. (Examples: $ sunny, in(alan, X) $).\\
\\
\textbf{Def:} A \red{definite clause} is either an atom (fact) or of the form:
\begin{align*}
h \leftarrow b_1 \land \dots \land b_m
\end{align*} 
where $ h $ and the $ b_i $ are atoms (Read this as "$ h $ if $ b $"). If $ m>0 $, the clause is called a rule. If $ m=0 $ the arrow can be omitted and the clause is called \textbf{atomic clause }or \textbf{fact}. An atomic clause has an \textbf{empty body}. Example:
\begin{align*}
in(X,Z) \leftarrow in(X,Y) \land part\_of(Y,Z)
\end{align*}
\textbf{Def:} A \red{knowledge base} is a set of definite clauses.\\
\\
\textbf{Def:} a \red{query} is of the form:
\begin{align*}
\mathbf{ask} \quad a_1 \land \dots \land a_m
\end{align*}
\subsection*{Datalog: Top Down Proof Procedure}
An extension of the top-down proof procedure can be applied to Datalog. The idea goes as follows:
\begin{itemize}[label = -]
	\item We find a clause in the $ KB $ whose $ head $ matches the query. 
	\item Substitute variables in the clause with their matching constants
\end{itemize}
For instance consider the following $ KB $ (we replace $ \land $ with \texttt{\&}):
\begin{lstlisting}
		in(alan, r123)
		part_of(r123, cs_building)
		in(X,Y) <- part_of(Z,Y) & in(X,Z)
\end{lstlisting}
The if our query is "\texttt{yes <- in(alan, cs_building)}", then using our knowledge base this translates to:
\begin{lstlisting}
		yes <- part_of(Z,cs_building) & in(alan,Z)
\end{lstlisting}
A full sketch of the proof is shown in the following slide:
\centerfig{0.9}{datalog-1}
\subsection*{Datalog: queries with variables}
Using our previous knowledge base, suppose now that our query is of the form:
\begin{lstlisting}
		in(alan, X1) 
		yes(X1) <- in(alan, X1)
\end{lstlisting}
What should the answer(s) be?
\textbf{NOT QUITE SURE GO OVER}
\end{document}